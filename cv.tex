\PassOptionsToPackage{dvipsnames}{xcolor}
\documentclass[10pt,a4paper]{altacv}

\usepackage[english]{babel}
\usepackage[hidelinks]{hyperref}
\usepackage[utf8]{inputenc}
\usepackage[T1]{fontenc}
\usepackage[default]{lato}
\usepackage{microtype}
\usepackage{fancyhdr}

\geometry{left=1cm,right=9cm,marginparwidth=6.8cm,marginparsep=1.2cm,top=1cm,bottom=0.5cm,headsep=15pt}

\definecolor{Dassault}{HTML}{154364}
\definecolor{SlateGrey}{HTML}{2E2E2E}
\definecolor{LightGrey}{HTML}{666666}
\colorlet{accent}{Dassault}
\colorlet{emphasis}{SlateGrey}
\colorlet{body}{LightGrey}

\renewcommand{\itemmarker}{{\small\textbullet}}
\renewcommand{\ratingmarker}{\faCircle}

\hypersetup{pdftitle={CV},pdfauthor={Tianyi Li},pdfkeywords={PhD, R\&D, multiphysics, numerical simulation, scientific computing, programming}}

\begin{document}
\name{Tianyi \underline{LI}}
\tagline{PhD, R\&D engineer in multiphysics, numerical simulation and scientific computing}
\photo{2.5cm}{photo.jpg}
\personalinfo{%
  \email{\href{mailto:tianyikillua@gmail.com}{tianyikillua@gmail.com}}
  \phone{\href{tel:+33646777690}{+33 6 46 77 76 90}}
  \location{\href{https://www.google.com/maps/place/12e+Arrondissement}{Paris 12e, FRANCE}}
  \\
  \linkedin{\href{https://www.linkedin.com/in/tianyikillua}{linkedin.com/in/tianyikillua}}
  \github{\href{https://github.com/tianyikillua}{github.com/tianyikillua}}
  \printinfo{\faQuestion}{\href{https://github.com/tianyikillua/cv/blob/master/cv.pdf}{check latest version of this CV}}
}

\pagestyle{fancy}
\fancyheadoffset{8cm}
\fancyhf{}
\renewcommand{\headrulewidth}{0pt}
\rhead{\scriptsize\color{body} updated on \today}

%% Make the header extend all the way to the right, if you want.
\begin{fullwidth}
\makecvheader
\end{fullwidth}

%% Depending on your tastes, you may want to make fonts of itemize environments slightly smaller
\AtBeginEnvironment{itemize}{\small}

\cvsection[cv_sidebar]{Experiences}

\cvevent{Research and Development Engineer\hfill\textcolor{body}{\normalfont\normalsize permanent (\emph{CDI})}}{\href{https://promold.fr}{Promold} \hfill\textcolor{body}{\normalfont \emph{TPE} -- consulting in simulation methods for plastics}}{Apr 2013 -- Aug 2013, Apr 2017 --}{Paris 17e, FRANCE}
\begin{itemize}
\item Fiber orientation modeling for process (injection molding) simulation of fiber-reinforced polymers with \textbf{Moldflow} and \textbf{Moldex3D}
\item Integrative structural analysis under \textbf{Optistruct} / \textbf{Radioss} / \textbf{code\_aster} with process-induced microstructural properties using \textbf{Digimat}
\item Improved multiscale rheological (fluid) and thermomechanical (solid) modeling of fiber-reinforced polymers: anisotropic viscosity, fiber orientation, structural buckling, porosity prediction and material failure behavior
\item Material model implementation for process simulation using \textbf{C++}, and for structural analysis using \textbf{UMAT} / \textbf{Fortran}
\item Uncertainty quantification and propagation for injection molding simulations using \textbf{OpenTURNS}
\item Development of various GUI-based simulation tools using \textbf{Python} / \textbf{C++}
  \begin{itemize}
    \item Implementing an integrative simulation methodology between process and product structural analysis
    \item Implementing a novel global adaptive optimization methodology of fiber orientation model parameters for a better correlation with experiment
    \item For buckling analysis of anisotropic fiber-reinforced materials (with finite element library \textbf{FEniCS} and eigenvalue solver \textbf{SLEPc})
  \end{itemize}
\item Development of scientific computing tools: procedure automation under \textbf{HyperWorks} using \textbf{TCL}; \textbf{Docker} deployment for launching simulations across systems; post-processing of simulation results under \textbf{ParaView} with \textbf{Python}; statistical data analysis and visualization under \textbf{Python/Jupyterlab}
\end{itemize}

\divider

\cvevent{Junior Research Engineer (PhD Candicate)\hfill\textcolor{body}{\normalfont\normalsize fixed term (\emph{CDD})}}{\href{http://www.imsia.cnrs.fr}{IMSIA (CNRS-EDF-CEA)} \hfill\textcolor{body}{\normalfont \emph{PME} -- applied research lab}}{Oct 2013 -- Sep 2016}{Palaiseau (91), FRANCE}
\begin{itemize}
\item Dynamic fracture modeling of brittle materials for concrete structures, with a novel non-local constitutive behavior for a better prediction and understanding of crack propagation behavior
\item Structural analysis, and model implementation in an industrial explicit dynamics finite element program \textbf{Europlexus} using \textbf{Fortran}
\item Design and implementation of parallel computing architecture using \textbf{MPI} and \textbf{PETSc} under \textbf{Europlexus}, quasi-perfect scaling efficiency achieved
\item Contributions to the open-source scientific computing libraries \textbf{FEniCS} and \textbf{PETSc} using \textbf{C++}
\end{itemize}

\divider

\cvevent{Structural Analysis Engineer\hfill\textcolor{body}{\normalfont\normalsize intern}}{\href{https://www.faurecia.com}{Faurecia Interior Systems} \hfill\textcolor{body}{\normalfont \emph{GE} -- automotive equipment supplier}}{Sep 2012 -- Feb 2013}{Méru (60), FRANCE}
\begin{itemize}
\item Elastoplastic constitutive modeling of long-fiber reinforced thermoplastics for the automobile industry, better agreement with experiment achieved
\item Numerical analysis and model implementation using \textbf{Python}
\item Static, modal and dynamic structural analysis under \textbf{Abaqus}
\end{itemize}

\divider

\cvevent{Mechanical Design Engineer \hfill\textcolor{body}{\normalfont\normalsize intern}}{\href{http://www.aml-systems.com}{AML-Systems} \hfill\textcolor{body}{\normalfont \emph{PME} -- automotive equipment supplier}}{Sep 2011 -- Feb 2012}{Le Bourget (93), FRANCE}
\begin{itemize}
\item Design and static analysis of headlamp cleaning systems using \textbf{Catia}
\item Analysis of experimental data using \textbf{Matlab}
\end{itemize}

\end{document}
